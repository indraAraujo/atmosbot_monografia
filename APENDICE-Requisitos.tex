\section{Introdução}
Esta seção apresentará o propósito do documento de Especificação de Requisitos do Sistema, o escopo do sistema, as definições de termos, as abreviações e acrônimos, ademais da organização do documento.

\subsection{Propósito}
A Especificação de Requisitos do Sistema expõe as necessidades do robô autônomo móvel simulado que o trabalho propõe, assim como as exigências do ambiente simulado no qual o robô irá atuar. Com isso, será facilitada a implementação do modelo a ser proposto, além de permitir a realização de testes específicos e coerentes ao sistema. 

\subsection{Escopo}
O sistema AtmosBot inicial modelado é um robô autônomo móvel que atua em um ambiente interno dinâmico desconhecido. O agente deve conseguir realizar a tarefa elementar de um robô de serviço doméstico completamente autônomo, sendo ela: a navegação autônoma. Com isso, o robô simulado modelado deve ser capaz de se locomover pelo ambiente de forma autônoma e sem colidir com possíveis obstáculos.

\subsection{Definições}

RFS - Requisitos Funcionais do Sistema

RNFS - Requisitos Não Funcionais do Sistema

RNFA - Requisitos Não Funcionais do Ambiente

\subsection{Organização}
Este documento contém a descrição geral do projeto, incluindo a perspectiva do sistema, suas funções e as características do usuário. Ademais, possui os requisitos específicos funcionais e não funcionais subdivididos entre o sistema e o ambiente simulados.

\section{Descrição geral}
Esta seção descreve os fatores gerais que afetam o sistema e seus requisitos, contendo a perspectiva do sistema, as suas funções e as características do usuário.

\subsection{Perspectiva do sistema}
O sistema em questão funciona de forma independente para ser um robô autônomo móvel simulado. Entretanto, em um futuro trabalho, ele deve ser inserido como funcionalidade de um sistema maior que é um robô de serviço doméstico. 

Como boa prática, segundo os princípios da arquitetura de subsunção de \citet{brooks85}, cada funcionalidade de um robô de serviço doméstico deve ser independente e apenas realizar comunicações necessárias entre si. Então, o sistema a ser modelado deve funcionar de forma modular do projeto completo.

\subsection{Funções do sistema}
O dispositivo simulado deve conter duas funções essenciais ao seu funcionamento, sendo elas:
\begin{enumerate}
    \item Navegação autônoma pelo ambiente sem colidir com obstáculos;
    \item Exploração do ambiente.
\end{enumerate}
Todas as sub-tarefas fundamentais das funções supracitadas devem ser cumpridas integralmente para o funcionamento correto do sistema.

\subsection{Características do usuário}
O sistema descrito é autônomo e não necessita da interação ou interferência de usuários humanos. Ele atua unicamente no ambiente inserido conforme as características do mesmo.

\section{Requisitos específicos}
Esta seção destaca com detalhe todos os requisitos funcionais e não funcionais do robô móvel autônomo simulado e do ambiente que será criado por simulação a fim de testar o sistema proposto. Os requisitos funcionais determinam as ações fundamentais que o sistema e o ambiente devem ter para que o dispositivo a ser modelado consiga aceitar e processar as suas entradas, além de gerar saídas para seus atuadores agirem no meio. 

Os requisitos não funcionais englobam questões de desempenho, segurança, confiabilidade, entre outros aspectos que são características necessárias para um ótimo funcionamento geral, não conceituando ações do sistema.

Todos os requisitos são rotulados pela prioridade e urgência de aplicação, podendo ser definidos como:
\begin{itemize}
    \item Essencial: o requisito não pode faltar no sistema ou ambiente;
    \item Importante: o requisito é necessário para um bom funcionamento, porém não o limita;
    \item Desejável: o requisito é necessário para um ótimo funcionamento, mas pode ser implementado em uma futura versão sem impactar o desempenho do sistema.
\end{itemize}
Além disso, os requisitos são expostos segundo a Tabela \ref{tab:modeloRequisitos}.

\begin{table}[H]
\centering
\caption{Modelo dos requisitos}
\label{tab:modeloRequisitos}
\resizebox{\textwidth}{!}{%
\begin{tabular}{l|p{15cm}|l}
\textbf{Identificação do requisito} & \textbf{Título do requisito}                    & \textbf{Prioridade}                   \\ \hline
\textbf{Entrada}                    & \multicolumn{2}{p{17cm}}{Especificação da entrada necessária para a funcionalidade}           \\ \hline
\textbf{Detalhamento}               & \multicolumn{2}{p{17cm}}{Detalhamento da funcionalidade}                                      \\ \hline
\textbf{Saída}                      & \multicolumn{2}{p{17cm}}{Especificação da saída necessária após a execução da funcionalidade} \\ \hline
\end{tabular}%
}
\caption*{Fonte: Autora (2023).}
\end{table}

\subsection{Requisitos Funcionais do Sistema}

\begin{table}[H]
\centering
\caption{RFS01}
\resizebox{\textwidth}{!}{%
\begin{tabular}{l|p{15cm}|l}
\textbf{RFS01} & \textbf{Movimentação do robô}                    & \textbf{Importante}                   \\ \hline
\textbf{Entrada}                    & \multicolumn{2}{p{17cm}}{Direção definida pelo eixo que o robô precisa se locomover.}           \\ \hline
\textbf{Detalhamento}               & \multicolumn{2}{p{17cm}}{O robô deve se mover para todas as direções no ambiente.}                                      \\ \hline
\textbf{Saída}                      & \multicolumn{2}{p{17cm}}{Movimentação do instrumento de locomoção (rodas ou pernas) do robô conforme a direção explicitada.} \\ \hline
\end{tabular}% 
}
\caption*{Fonte: Autora (2023).}
\end{table}

\begin{table}[H]
\centering
\caption{RFS02}
\resizebox{\textwidth}{!}{%
\begin{tabular}{l|p{15cm}|l}
\textbf{RFS02} & \textbf{Evitação de obstáculos}                    & \textbf{Essencial}                   \\ \hline
\textbf{Entrada}                    & \multicolumn{2}{p{17cm}}{Dados do ambiente ao redor do robô coletados a partir de sensores em seu corpo.}           \\ \hline
\textbf{Detalhamento}               & \multicolumn{2}{p{17cm}}{O robô deve processar os dados dos sensores de modo a saber se há um obstáculo (objeto, pessoa, animal, parede ou móvel) perto o suficiente que ele possa colidir. Com essa informação, caso haja um obstáculo próximo a sua frente, o robô deve se esquivar dele e continuar sua trajetória planejada. Caso não haja obstáculos a sua frente, o robô deve continuar a sua trajetória planejada.}                                      \\ \hline
\textbf{Saída}                      & \multicolumn{2}{p{17cm}}{Comandos para os motores atuarem nas rodas alterando a direção conforme o obstáculo detectado.} \\ \hline
\end{tabular}% 
}
\caption*{Fonte: Autora (2023).}
\end{table}


\begin{table}[H]
\centering
\caption{RNFS01}
\resizebox{\textwidth}{!}{%
\begin{tabular}{l|p{15cm}|l}
\textbf{RNFS01} & \textbf{Locomoção em diferentes superfícies}                    & \textbf{Importante}                   \\ \hline
\textbf{Entrada}                    & \multicolumn{2}{p{17cm}}{Não possui.}           \\ \hline
\textbf{Detalhamento}               & \multicolumn{2}{p{17cm}}{O robô deve ser capaz de se locomover de forma satisfatória, tanto em pisos lisos,  quanto em pisos revestidos (com carpetes ou tapetes).}                                      \\ \hline
\textbf{Saída}                      & \multicolumn{2}{p{17cm}}{Não possui.} \\ \hline
\end{tabular}% 
}
\caption*{Fonte: Autora (2023).}
\end{table}

\begin{table}[H]
\centering
\caption{RNFS02}
\resizebox{\textwidth}{!}{%
\begin{tabular}{l|p{15cm}|l}
\textbf{RNFS02}& \textbf{Locomoção com segurança aos seres transeuntes}                    & \textbf{Desejável}                   \\ \hline
\textbf{Entrada}                    & \multicolumn{2}{p{17cm}}{Não possui.}           \\ \hline
\textbf{Detalhamento}               & \multicolumn{2}{p{17cm}}{A velocidade de locomoção do robô deve ser moderada, respeitando o ambiente interno e sua dinamicidade, a fim de não prejudicar as pessoas e/ou animais que transitam o espaço em conjunto com o próprio robô. Sendo assim, ela não deve ultrapassar de 0,2 m/s. }                                      \\ \hline
\textbf{Saída}                      & \multicolumn{2}{p{17cm}}{Não possui.} \\ \hline
\end{tabular}% 
}
\caption*{Fonte: Autora (2023).}
\end{table}


\subsection{Requisitos Não Funcionais do Ambiente Simulado}


\begin{table}[H]
\centering
\caption{RNFA01}
\resizebox{\textwidth}{!}{%
\begin{tabular}{l|p{15cm}|l}
\textbf{RNFA01} & \textbf{Dinamicidade}                    & \textbf{Essencial}                   \\ \hline
\textbf{Entrada}                    & \multicolumn{2}{p{17cm}}{Não possui.}           \\ \hline
\textbf{Detalhamento}               & \multicolumn{2}{p{17cm}}{O ambiente deve conter mudanças frequentes na posição de objetos e móveis.}                                      \\ \hline
\textbf{Saída}                      & \multicolumn{2}{p{17cm}}{Não possui.} \\ \hline
\end{tabular}% 
}
\caption*{Fonte: Autora (2023).}
\end{table}


\begin{table}[H]
\centering
\caption{RNFA02}
\resizebox{\textwidth}{!}{%
\begin{tabular}{l|p{15cm}|l}
\textbf{RNFA02} & \textbf{Semelhança a domicílios reais}                    & \textbf{Essencial}                   \\ \hline
\textbf{Entrada}                    & \multicolumn{2}{p{17cm}}{Não possui.}           \\ \hline
\textbf{Detalhamento}               & \multicolumn{2}{p{17cm}}{O ambiente deve trazer aspectos de um domicílio comum, com móveis, paredes, portas, objetos, tapetes, entre outros itens que o tornam mais verossímil à realidade.}                                      \\ \hline
\textbf{Saída}                      & \multicolumn{2}{p{17cm}}{Não possui.} \\ \hline
\end{tabular}% 
}
\caption*{Fonte: Autora (2023).}
\end{table}

\begin{table}[H]
\centering
\caption{RNFA03}
\resizebox{\textwidth}{!}{%
\begin{tabular}{l|p{15cm}|l}
\textbf{RNFA03} & \textbf{Tamanho de domicílio padrão}                    & \textbf{Desejável}                   \\ \hline
\textbf{Entrada}                    & \multicolumn{2}{p{17cm}}{Não possui.}           \\ \hline
\textbf{Detalhamento}               & \multicolumn{2}{p{17cm}}{O ambiente deve conter, no mínimo, três cômodos internos que se conectam de alguma forma (por portas ou corredores).}                                      \\ \hline
\textbf{Saída}                      & \multicolumn{2}{p{17cm}}{Não possui.} \\ \hline
\end{tabular}% 
}
\caption*{Fonte: Autora (2023).}
\end{table}

\begin{table}[H]
\centering
\caption{RNFA04}
\resizebox{\textwidth}{!}{%
\begin{tabular}{l|p{15cm}|l}
\textbf{RNFA04}& \textbf{Minimização de Impedimentos}                    & \textbf{Essencial}                   \\ \hline
\textbf{Entrada}                    & \multicolumn{2}{p{17cm}}{Não possui.}           \\ \hline
\textbf{Detalhamento}               & \multicolumn{2}{p{17cm}}{O ambiente não deve conter portas fechadas ou pouco abertas. Além disso o ambiente de acesso do robô não deve conter escadas ou degraus altos (devem conter no máximo 1 centímetro de altura).}                                      \\ \hline
\textbf{Saída}                      & \multicolumn{2}{p{17cm}}{Não possui.} \\ \hline
\end{tabular}% 
}
\caption*{Fonte: Autora (2023).}
\end{table}
