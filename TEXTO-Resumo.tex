
\begin{abstract}
Os robôs de serviço doméstico estão cada vez mais presentes, podendo desempenhar um papel relevante no auxílio da execução das tarefas diárias domiciliares, em diversas ocasiões. Eles são benéficos para as pessoas com algum grau de limitação motora, sobrecarga laboral, famílias monoparentais ou até mesmo como regalia. Entretanto, o processo completo de elaborar, desenvolver e validar um robô de serviço doméstico, consome recursos e tempo em excesso. Esses gastos contrapõem-se  às teorias de \textit{time-to-market} e do triângulo de ferro que definem a atual dinâmica do mercado e de inovações. Dito isso, para contornar essa problemática, neste trabalho é proposto a modularização do desenvolvimento de um robô de serviço doméstico, implementando funcionalidades independentes com as tecnologias mais relevantes e adequadas. Assim, foi modelado um robô simulado com a capacidade de executar a funcionalidade essencial do robô de serviço doméstico: a navegação autônoma. Foi utilizada a abordagem SLAM em combinação com sensor LiDaR para mapear, localizar e, consequentemente, navegar em um ambiente domiciliar desconhecido simulado. No intuito de validar o modelo desenvolvido, uma série de testes foi efetuada conforme os cenários estabelecidos em um conjunto de casos de teste, elaborados para este trabalho. Ademais, foi realizada uma análise comparativa entre trabalhos correlatos e os resultados obtidos com o AtmosBot. Por fim, com tal validação foi identificado que o modelo proposto apresenta o comportamento esperado para um robô autônomo móvel doméstico, utilizando tecnologias relevantes ao contexto inserido, além de adequadas para redução de recursos e tempo dispendidos no seu desenvolvimento.
\end{abstract}

\begin{englishabstract}{Navigation, Exploration, Simulation}
Domestic service robots are increasingly present and can play a relevant role in helping people carry out daily household tasks on a number of occasions. They are beneficial for people with some degree of motor limitation, work overload, single-parent families, or even as a perk. However, the entire process of designing, developing and validating a domestic service robot consumes excessive resources and time. These costs are contrary to the theories of time-to-market and the iron triangle that define the current market and innovation dynamic. In order to overcome this problem, this work proposes modularizing the development of a domestic service robot, implementing independent functionalities with the most relevant and appropriate technologies. Thus, a simulated robot was modeled with the ability to perform the essential functionality of a domestic service robot: the autonomous navigation. The SLAM approach was used in combination with the LiDaR sensor to map, locate and, consequently, navigate in an unknown simulated home environment. In order to validate the model developed, a series of tests were carried out according to the scenarios established in a set of test cases prepared for this work. In addition, a comparative analysis was carried out between related work and the results obtained with AtmosBot. Finally, this validation identified that the proposed model presents the expected behavior for an autonomous mobile domestic robot, using technologies that are relevant to the context in which it is inserted, as well as being suitable for reducing resources and time spent on its development.
\end{englishabstract}
