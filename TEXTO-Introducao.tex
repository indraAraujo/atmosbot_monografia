A execução de tarefas consideradas básicas do cotidiano, como preparar uma mesa de jantar; apanhar uma caneca no armário; colocar roupas na máquina de lavar, pode ser impactada por uma variedade de razões. No âmbito da saúde, a limitação motora, seja por um grau leve ou grave, afeta diretamente as atividades do dia a dia. 

\citet{strokeStatistics:2022} indica, conforme uma análise dos dados coletados pelo\textit{ Global Burden of Disease}, que o Acidente Vascular Cerebral (AVC) é a terceira principal causa de deficiência, com um aumento significativo de aproximadamente 143\% desde 1990 até 2019. Em 80\% dos sobreviventes de  AVC é apresentado algum grau de deficiência motora, como a fraqueza dos membros \cite{postStroke:2019}. 

O AVC não é a única causa para limitações motoras, também existem doenças neuromusculares e congênitas, como esclerose lateral amiotrófica e paralisia cerebral, que afetam 0,44\% da população mundial  e 0,16\% da população nos países de alta renda,  respectivamente \cite{alsStatistics:2020,palsyStatistcs:2022}. Esses casos não são restritos a doenças, podendo ser causados até mesmo pelo o avanço da idade \cite{elderlyMobility:2019}. 

Ter algum tipo de deficiência motora, temporária ou permanente, é algo que afeta (ou afetará) praticamente todos os seres humanos, independente da nacionalidade, etnia ou outra característica  \cite{omsDisability:2023}. Segundo \citet{omsDisability:2023}, cerca de 16\% da população mundial tem uma deficiência significativa, tal estimativa corresponde a 13 milhões de pessoas no mundo atualmente. Apesar disso, o amparo a essas dificuldades não é potencializado na sociedade. 

As residências atuais possuem impedimentos para aqueles com limitações motoras por conta das suas arquiteturas inacessíveis \cite{whoHousing:2018}. Não só isso: a logística do \textit{Home Care}, que conceitua o atendimento médico por profissionais da saúde diretamente no conforto do domicílio das pessoas, muitas vezes, é incapacitada na região onde a pessoa em necessidade se encontra; assim como, frequentemente não é possível ter o amparo social, da família e dos amigos \cite{homeCare:2018}. 

Entretanto, uma alternativa que cresce e fortalece com o avanço da tecnologia nas últimas décadas são os robôs autônomos móveis, capazes de executar tarefas cotidianas em uma casa de acordo com comandos de um ser humano. Um exemplo comumente encontrado e amplamente utilizado na atualidade são os aspiradores de pó autônomos, que desempenham um papel importante ao facilitar as atividades diárias, como a higienização básica do ambiente de convívio \cite{roombaSite}. 

O protótipo \textit{Handy} da Samsung, demonstrado em 2021 por uma coletiva de imprensa, é outro robô que serve perfeitamente para o auxílio de pessoas com limitação motora \cite{pressHandy}. Ele desempenha um papel semelhante ao de um empregado,  movendo objetos de um ponto a outro conforme um comando, arrumando a mesa de jantar ou  até mesmo colocando as louças para lavar. 

Os robôs autônomos móveis são dispositivos que conseguem navegar em um espaço sem monitoramento a fim de executarem uma atividade determinada. Regularmente, são utilizados no contexto de robôs de serviço doméstico, os quais conseguem interagir com seres humanos, receber comandos e a partir deles, atuar no ambiente. A base desses robôs é composta por funcionalidades essenciais, cada qual com seus desafios específicos, sendo elas: i) navegação autônoma; ii) reconhecimento de objetos; iii) manipulação de objetos e iv) interação robô-humano.

Como supracitado, os robôs autônomos móveis, em específico os de serviço doméstico, são dispositivos complexos com sub-tarefas cruciais para o seu funcionamento. Modelar, construir e testar estes sistemas são trabalhos excessivamente extensos e complicados, necessitando recursos, tempo e uma equipe de profissionais. Cada funcionalidade necessita ser testada exaustivamente, individual e integralmente com o sistema, a fim de captar falhas e possibilitar aprimoramento.

Ao desenvolver uma nova aplicação, se torna desvantajosa a presença dos fatores mencionados  (alta complexidade e extensão). Visto que, no âmbito empresarial da tecnologia, há uma discussão em relação ao \textit{time-to-market}, tornando esses elementos uma preocupação para o projeto. Isso é resultado da exigência criada pelo mercado para que o lançamento de um novo produto ocorra no momento certo \cite{ttmTradeOff:2021,npdTTM:1996, ttmConcurrent:2011}. Portanto, o desenvolvimento de um novo artefato pode ser favorecido pela redução do tempo da sua elaboração, tanto para os aspectos de relevância no mercado quanto de custos \cite{npdTTM:1996, ttmConcurrent:2011}.

Ademais, a produção de um novo artefato, como um robô, é o fruto da execução de um projeto. Sendo assim, ao desenvolver um projeto, é essencial manter o equilíbrio entre o escopo, tempo e os recursos utilizados, para que a sua qualidade seja assegurada \cite{pmbok}. Esses fatores são pilares interdependentes, identificados como o triângulo de ferro de um projeto, que norteiam o resultado do produto final \cite{pmbok}. Ao utilizar meios para desenvolver um robô de serviço doméstico, por exemplo, em um tempo adequado e consumindo apenas os recursos necessários, asseguramos que os seus objetivos e requisitos (o escopo) sejam alcançados de forma satisfatória \cite{pmbok}.  

A partir da problemática apresentada pelas teorias de \textit{time-to-market} e do triângulo de ferro, 
seria possível desenvolver um robô de serviço doméstico com as tecnologias adequadas e relevantes?

Diante desse problema, um caminho satisfatório para contornar tais dificuldades é a modelagem simulada das tarefas do robô. Inicialmente, as funcionalidades mais essenciais do sistema devem ser moduladas, a fim de obter uma base íntegra e estável para futuros incrementos e aprimoramentos. Portanto, em suma, este projeto propõe um modelo incrementável simulado de um robô capaz de realizar a atividade primordial de um robô de serviço doméstico: a navegação autônoma.

\section{Objetivos}
\label{sec-objetivos}

\subsection{Objetivo Geral}

Este trabalho visa propor um modelo simulado de um robô autônomo móvel doméstico. O robô da presente proposta tem o objetivo de atuar simuladamente em um ambiente interno dinâmico desconhecido e realizar a tarefa primordial de um robô de serviço doméstico: a navegação autônoma.

\subsection{Objetivos Específicos}
\begin{itemize}
  \item  Analisar tecnologias variadas relevantes de robôs móveis autônomos com propostas semelhantes por meio de uma revisão bibliográfica narrativa;
  \item Dentre as tecnologias encontradas, selecionar as que apresentam maior relevância conforme análise efetuada, a fim de implementá-las para realizarem a navegação autônoma do robô em um ambiente interno dinâmico desconhecido;
  \item Validar o modelo proposto do robô autônomo móvel simulado.
\end{itemize}

\section{Organização do trabalho}
\label{sec-organizacao}

A fim de expor o conhecimento necessário deste trabalho, o mesmo foi organizado com o \chapterautorefname~\ref{cap-metodologia}, no qual foram explicadas as etapas da metodologia utilizada. Em seguida, há os capítulos \chapterautorefname~\ref{cap-revisao-bibliografica} e \chapterautorefname~\ref{cap-trabalhos-relacionados} que fundamentam as teorias de robôs autônomos móveis e simulação robótica, além de exporem as pesquisas mais recentes e relevantes correlatas ao AtmosBot.  Posteriormente, se encontra o \chapterautorefname~\ref{cap-desenvolvimento} com a proposta, a modelagem do protótipo e os detalhes do desenvolvimento da solução. Por fim, há o \chapterautorefname~\ref{cap-resultadosdiscussao}, onde são expostos e discutidos os resultados obtidos, além do \chapterautorefname~\ref{cap-consideracoesFinais} com as considerações finais.
Ademais, o trabalho possui o \appendixautorefname~\ref{appendix-requisitos} que constitui a especificação dos requisitos do sistema e do ambiente simulados, além do \appendixautorefname~\ref{appendix-casosTeste} com os casos de teste utilizados para a validação final do modelo e o \appendixautorefname~\ref{appendix-resultadosTestes}, expondo os resultados obtidos nos testes executados. 
