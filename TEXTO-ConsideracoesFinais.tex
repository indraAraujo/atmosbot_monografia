
Este trabalho trouxe à tona o questionamento da possibilidade de desenvolver um robô de serviço doméstico com as tecnologias adequadas e relevantes, perante o atual momento caracterizado pelas teorias de  \textit{time-to-market} e do triângulo de ferro. Assim, foi identificado que, para esse desenvolvimento, é necessário modularizar o robô de serviço doméstico em funcionalidades independentes, se beneficiando de testes simulados para minimizar os recursos financeiros e o tempo despendido.


Dito isso, o presente trabalho conseguiu alcançar o seu objetivo geral definido com sucesso, sendo este elaborar um modelo de robô autônomo móvel simulado, capaz de realizar a funcionalidade fundamental de um robô de serviço doméstico: a navegação autônoma. O AtmosBot é formado por uma estrutura física com quatro rodas e um sensor LiDaR. Com as informações do sensor e da odometria das rodas, o robô tem a capacidade de mapear e se localizar no ambiente simulado desconhecido com abordagem SLAM ii) planejar e prosseguir por uma trajetória até um ponto de destino de forma autônoma sem colidir com obstáculos.

O desenvolvimento do modelo proposto foi possível a partir de uma revisão bibliográfica narrativa que fundamentou as teorias necessárias para a navegação autônoma e simulação robótica. Com as pesquisas bibliográficas constituídas por critérios pré-definidos, de inclusão e exclusão, foi possível analisar as tecnologias relevantes para a implementação de um robô autônomo móvel, alcançado o primeiro objetivo específico deste trabalho. Ademais, com os resultados dessas pesquisas bibliográficas específicas
foi encontrado que a abordagem SLAM é a mais implementada para a localização de um robô e o sensor LiDaR é o instrumento mais utilizado para a percepção de ambientes. Com isso,  foi implementado o SLAM e o sensor LiDaR para executar a navegação autônoma do robô, permitindo alcançar o segundo objetivo específico do presente trabalho. Por fim, o modelo integrado foi validado com sucesso por testes realizados conforme os casos de teste elaborados, expondo que todos os requisitos levantados foram atingidos. Além disso, foi realizada uma análise comparativa com os trabalhos correlatos encontrados, demonstrando que o modelo desenvolvido se comporta similarmente com propostas do mesmo âmbito, com uma redução de tempo e recursos despendidos. Portanto, destaca-se que todos os objetivos específicos definidos também foram alcançados com sucesso.

Confirmando a possibilidade de desenvolver moduladamente um robô de serviço doméstico com uma redução de recursos despendidos, conforme comprovado por este trabalho, é possível extrapolar tal ideia e considerar a probabilidade de tornar esses robôs mais acessíveis economicamente. Com isso, pode-se vislumbrar  que futuramente eles se tornem presentes até nas casas de pessoas de baixa renda com limitações motoras, auxiliando-as nas suas tarefas diárias e aumentando a sua qualidade de vida. 

Inicialmente foi idealizado o desenvolvimento de um modelo de robô autônomo móvel com capacidade de navegar autonomamente em direção a um objeto detectado no ambiente. As tecnologias mais relevantes no tema do trabalho seriam comparadas, por uma sequência de testes, a fim de identificar as ferramentas mais adequadas para cada sub-tarefa presente no sistema. Para condizer com o trabalho de conclusão de curso, o escopo foi reduzido para um robô autônomo móvel doméstico implementado com tecnologias selecionadas por uma análise de trabalhos correlatos. Além disso, o modelo implementa suas atuais funcionalidades de forma independente, podendo suportar incrementos de outras tarefas importantes para um robô de serviço doméstico. Diante disso, os requisitos do sistema foram identificados novamente para melhor suportar a solução proposta.

Dessa forma, este trabalho propõe um robô autônomo móvel doméstico independente capaz de ser integrado em um sistema maior para compor um robô de serviço doméstico. Assim, se vê como futuros trabalhos, a implementação de um módulo de localização e manipulação de objetos. Além disso, se torna viável a integração de uma interface para interação com humanos, possibilitando receber ordens de uma pessoa e executar atividades específicas em um ambiente, com o intuito de auxiliar nas suas tarefas diárias.