\section{Introdução}
Este apêndice tem o intuito de explicar e expor os casos de teste definidos para realizar a validação final do modelo proposto para um robô autônomo móvel perante os requisitos previamente definidos para o sistema completo (robô e ambiente simulados).

\section{Casos de teste}
Os casos de teste são procedimentos para realizar testes nos quais são dispostos as ações necessárias para realizar cada teste e o resultado ideal perante estas ações.
Para melhor organização e padronização, os casos de teste aqui definidos seguem o modelo disposto na Tabela  \ref{tab:modeloCasos}.

\begin{table}[H]
\centering
\caption{Modelo dos casos de teste}
\label{tab:modeloCasos}
\resizebox{\textwidth}{!}{%
\begin{tabular}{p{3cm}|p{5cm}|p{5cm}}
\multicolumn{1}{p{3cm}|}{\textbf{Requisito referente}} &
  \multicolumn{1}{p{5cm}|}{\textbf{Ação/Entrada}} &
  \multicolumn{1}{p{5cm}}{\textbf{Resultado esperado}} \\ \hline
Título do requisito ao que o caso de teste se refere. &
  Detalhamento da ação, ou entrada, necessária para validar o requisito. &
  Detalhamento do Resultado esperado da simulação perante a ação, ou entrada, definida. \\ \hline
\end{tabular}%
}
\caption*{Fonte: Autora (2023).}
\end{table}

A seguir são dispostos os casos de teste elaborados conforme os requisitos (funcionais e não funcionais) definidos previamente para a simulação do ambiente e do robô, expostos no \appendixautorefname~\ref{appendix-requisitos}.

\begin{table}[p]
\centering
\caption{Caso de teste CT01 referente a RFS01 }
\label{tab:caso01}
\resizebox{\textwidth}{!}{%
\begin{tabular}{p{3cm}|p{5cm}|p{5cm}}
\multicolumn{1}{p{3cm}|}{\textbf{Requisito referente}} &
  \multicolumn{1}{p{5cm}|}{\textbf{Ação/Entrada}} &
  \multicolumn{1}{p{5cm}}{\textbf{Resultado esperado}} \\ \hline
Movimentação do robô &
  Em primeiro lugar, a simulação deve ser executada e o estado de vagar pelo ambiente deve ser inciado. Com tudo executando devidamente, devem ser adicionados obstáculos há 0,5 metros da dianteira e lateral esquerda do robô. Após 5 segundos, devem ser adicionados obstáculos há 0,5 metros da dianteira e lateral direita do robô. &
  Ao adicionar os primeiros obstáculos, o robô deverá se rotacionar no sentido direito. Após adicionar os últimos obstáculos, o robô deverá se rotacionar no sentido esquerdo.
  \\ \hline
\end{tabular}%
}
\caption*{Fonte: Autora (2023).}
\end{table}


\begin{table}[p]
\centering
\caption{Caso de teste CT02 referente a RFS02 }
\label{tab:caso02}
\resizebox{\textwidth}{!}{%
\begin{tabular}{p{3cm}|p{5cm}|p{5cm}}
\multicolumn{1}{p{3cm}|}{\textbf{Requisito referente}} &
  \multicolumn{1}{p{5cm}|}{\textbf{Ação/Entrada}} &
  \multicolumn{1}{p{5cm}}{\textbf{Resultado esperado}} \\ \hline
Evitação de obstáculos &
  Em primeiro lugar, a simulação deve ser executada e o estado de exploração do ambiente deve ser inciado. A exploração deve ocorrer por 120 segundos. &
  Com o ambiente propriamente elaborado, o robô deverá se locomover por ele e se esquivar de possíveis móveis que estão presentes no meio.
  \\ \hline
\end{tabular}%
}
\caption*{Fonte: Autora (2023).}
\end{table}

\begin{table}[p]
\centering
\caption{Caso de teste CT03 referente a RNFS01 }
\label{tab:caso04}
\resizebox{\textwidth}{!}{%
\begin{tabular}{p{3cm}|p{5cm}|p{5cm}}
\multicolumn{1}{p{3cm}|}{\textbf{Requisito referente}} &
  \multicolumn{1}{p{5cm}|}{\textbf{Ação/Entrada}} &
  \multicolumn{1}{p{5cm}}{\textbf{Resultado esperado}} \\ \hline
Locomoção em diferentes superfícies  &
  Em primeiro lugar, a simulação deve ser executada e o estado de vagar pelo ambiente deve ser inciado em um meio constituído parcialmente por piso liso e por tapete. &
  O robô deverá se movimentar igualmente em ambos as superfícies, sem que o robô apresente o comportamente de instabilidade nas rodas ou emperramento.
  \\ \hline
\end{tabular}%
}
\caption*{Fonte: Autora (2023).}
\end{table}

\begin{table}[p]
\centering
\caption{Caso de teste CT04 referente a RNFS02 }
\label{tab:caso05}
\resizebox{\textwidth}{!}{%
\begin{tabular}{p{3cm}|p{5cm}|p{5cm}}
\multicolumn{1}{p{3cm}|}{\textbf{Requisito referente}} &
  \multicolumn{1}{p{5cm}|}{\textbf{Ação/Entrada}} &
  \multicolumn{1}{p{5cm}}{\textbf{Resultado esperado}} \\ \hline
Locomoção com segurança aos seres transeuntes  &
  Em primeiro lugar, a simulação deve ser executada e o estado de exploração do ambiente deve ser inciado em um meio constituído por no mínimo um ser humano se locomovendo. &
  O robô deverá se locomover pelo ambiente em uma velocidade até 0,2 m/s.
  \\ \hline
\end{tabular}%
}
\caption*{Fonte: Autora (2023).}
\end{table}


\begin{table}[p]
\centering
\caption{Caso de teste CT05 referente a RNFA01 }
\label{tab:caso07}
\resizebox{\textwidth}{!}{%
\begin{tabular}{p{3cm}|p{5cm}|p{5cm}}
\multicolumn{1}{p{3cm}|}{\textbf{Requisito referente}} &
  \multicolumn{1}{p{5cm}|}{\textbf{Ação/Entrada}} &
  \multicolumn{1}{p{5cm}}{\textbf{Resultado esperado}} \\ \hline
Dinamicidade  &
  Em primeiro lugar, a simulação deve ser executada e a posição dos objetos e móveis devem ser alteradas com constância.   &
  Os objetos e móveis deverão ter sua posição alterada sem que a simulação pare ou corrompa. 
  \\ \hline
\end{tabular}%
}
\caption*{Fonte: Autora (2023).}
\end{table}


\begin{table}[p]
\centering
\caption{Caso de teste CT06 referente a RNFA03 }
\label{tab:caso09}
\resizebox{\textwidth}{!}{%
\begin{tabular}{p{3cm}|p{5cm}|p{5cm}}
\multicolumn{1}{p{3cm}|}{\textbf{Requisito referente}} &
  \multicolumn{1}{p{5cm}|}{\textbf{Ação/Entrada}} &
  \multicolumn{1}{p{5cm}}{\textbf{Resultado esperado}} \\ \hline
Semelhança a domicílios reais  &
  A simulação deve ser executada.   &
  A simulação deverá apresentar uma planta residencial básica com elementos de sala, dormitório e cozinha.
  \\ \hline
\end{tabular}%
}
\caption*{Fonte: Autora (2023).}
\end{table}

\begin{table}[p]
\centering
\caption{Caso de teste CT07 referente a RNFA04 }
\label{tab:caso10}
\resizebox{\textwidth}{!}{%
\begin{tabular}{p{3cm}|p{5cm}|p{5cm}}
\multicolumn{1}{p{3cm}|}{\textbf{Requisito referente}} &
  \multicolumn{1}{p{5cm}|}{\textbf{Ação/Entrada}} &
  \multicolumn{1}{p{5cm}}{\textbf{Resultado esperado}} \\ \hline
Tamanho de domicílio padrão  &
  A simulação deve ser executada.   &
  O ambiente simulado deverá apresentar no mínimo um espaço de cozinha, sala e dormitório. Entre os espaços, não devem se apresentar portas de acesso fechadas.
  \\ \hline
\end{tabular}
%
}
\caption*{Fonte: Autora (2023).}
\end{table}

\begin{table}[p]
\centering
\caption{Caso de teste CT08 referente a RNFA05 }
\label{tab:caso11}
\resizebox{\textwidth}{!}{%
\begin{tabular}{p{3cm}|p{5cm}|p{5cm}}
\multicolumn{1}{p{3cm}|}{\textbf{Requisito referente}} &
  \multicolumn{1}{p{5cm}|}{\textbf{Ação/Entrada}} &
  \multicolumn{1}{p{5cm}}{\textbf{Resultado esperado}} \\ \hline
Minimização de Impedimentos  &
  A simulação deve ser executada.   &
  O ambiente simulado não deverá apresentar portas de acesso, entre os espaços, que se apresentam fechadas.
  \\ \hline
\end{tabular}
%
}
\caption*{Fonte: Autora (2023).}
\end{table}